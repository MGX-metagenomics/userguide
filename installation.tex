
\chapter{Installation}
\label{installation}

\section{Prerequisites}

MGX is a client/server application based on the Netbeans Platform and implemented
in Java. To run the client application, the following dependencies have to be
met:

\begin{itemize}
  \item{Java Development Kit, 1.8 or later}
  \item{Supported Operating Systems: Windows, Linux, Mac OS X}
  \item{Memory requirements: $>=$ 2GB RAM}
  \item{(preferably broadband) Internet connection}
\end{itemize}

\subsection{Java}

MGX requires a working Java Development Kit (JDK) to be installed on the computer. Typically,
Java is already installed on most systems or can be obtained free of charge from\\

\url{http://www.oracle.com/technetwork/java/javase/downloads/jdk8-downloads-2133151.html}.\\

The installed Java version can be checked e.g. at\\

\url{http://www.java.com/de/download/installed.jsp}.

\subsection{Supported Operating Systems}

The MGX application is developed and regularly used on Unix-based systems,
but has already successfully been used on computers running Windows and
Mac OS X.

\subsection{Memory and disk requirements}

512MB of available main memory are sufficient to run MGX. Installation of the
software requires about 60MB of disk space.

\subsection{Internet connection}

The network communication protocol used by the MGX framework has been heavily optimized to
allow usage even with low-throughput connections. Thus, typical usage of the application
like visualization of analysis results does not require a lot of bandwidth, although
overall performance may suffer with high-latency or low-bandwidth connections. However, as
sequence datasets obtained by metagenome sequencing tend to be quite large, a broadband 
connection is recommended at least for initial data upload to the MGX server or when exporting
sequence data from a MGX project.

\subsection{Obtaining MGX}

We regularly publish new releases of the MGX client application, which are
available for download at\\

    \url{https://github.com/MGX-metagenomics/MGX-gui/releases}.\\

An installation isn't necessary, just unzip the file and start the software
from the \texttt{bin/} subdirectory (Linux: \texttt{mgx\_gui}; Windows: \texttt{mgx\_gui64.exe}). Please check whether an updated version
is available before reporting bugs.

